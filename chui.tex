\documentclass{ctexart}
\usepackage{tipa}
\usepackage{vowel}
\title{崔世安的普通話}
\author{奶茶貓}
\begin{document}
\maketitle
\tableofcontents

崔世安、崔世昌兩兄弟之官話發音經常于電視、網絡等媒體体被人詬病

\section{崔氏官話音素表}

\subsection{輔音}
崔氏官話中,/\textipa{\textctc}/ 與 /\textipa{s}/ 爲自由變體。

/\textipa{S}/ 於老派廣州話和其他粵語分支未併入/s/。在崔氏官話中,/\textipa{S}/ 與 /\textipa{\textrtails}/ 爲自由變體。同樣道理,/\textipa{ts}/ 與 /\textipa{tS}/、/\textipa{ts\super h}/ 與 /\textipa{tS\super h}/ 亦爲自由變體。崔拼中不二分,以zh、ch、sh表示。

/\textipa{\textrtailz}/ /\textipa{\:R}/

\subsection{元音}
崔氏元音主要取自普通話和沃門粵語的元音位置
\begin{figure}[h]
  \caption{崔氏官話元音}
    {\large\begin{vowel}
      \putcvowel[l]{i}{1}
      \putcvowel[l]{e}{2}
      \putcvowel[l]{\textepsilon}{3}
      \putcvowel[r]{\oe}{3}
      \putcvowel[l]{a}{4}
      \putcvowel[r]{\textscoelig}{4}
      \putcvowel[l]{\textscripta}{5}
      \putcvowel[r]{\textturnscripta}{5}
      \putcvowel[l]{\textturnv}{6}
      \putcvowel[r]{\textopeno}{6}
      \putcvowel[l]{\textramshorns}{7}
      \putcvowel[r]{o}{7}
      \putcvowel[l]{\textturnm}{8}
      \putcvowel[r]{u}{8}
      \putcvowel[l]{\textbari}{9}
      \putcvowel[r]{\textbaru}{9}
      \putcvowel[r]{\textbaro}{10}
      \putcvowel{\textschwa}{11}
      \putcvowel{\textturna}{15}
    \end{vowel}}
\end{figure}
\section{崔拼方案}

爲簡易
\end{document}
